\begin{resumo}
    
	O conceito \textit{Internet} das Coisas é o próximo ramo de estudo da computação, exigindo sistemas computacionais em objetos e sua conexão com a \textit{Internet}.
	A abordagem da pesquisa traz a caracterização da \textit{Internet} das Coisas e plataforma \textit{Java Embedded} para diferentes escopos de aplicações, como interfaceamento de periféricos de baixo nível, arquitetura cliente/servidor e processamento de eventos complexos.
	A metodologia aplicada com a plataforma \textit{Java Embedded} em aplicações de diferentes níveis de arquitetura demonstra o potencial da tecnologia.
	Pesquisar uma plataforma aplicável para diversos contextos, enriquece o desenvolvimento de novas tecnologias e traz novas possibilidades para o mercado, direcionando para futuros estudos em especificação e desenvolvimento de protocolos e sistemas operacionais especializados.
	
	\vspace{\onelineskip}
	\noindent
	\textbf{\uppercase{Palavras-chave}}: \palavraschave
    
\end{resumo}

\begin{resumo}[Abstract]
	\begin{otherlanguage*}{english}
		The Internet of Things concept is the next computing branch of study, requiring computer systems on objects and their connection to the Internet.
		The research approach brings the characterization of the Internet of Things and Embedded Java platform for different scopes of applications such as low-level peripheral interfacing, client / server architecture and complex event processing.
		The methodology applied to Embedded Java platform applications of different levels of architecture demonstrates the potential of technology.
		Search an applicable platform for different contexts enriches the development of new technologies and brings new possibilities to the market, directing to future studies on specification and development of protocols and specialized operating systems.
				
		\vspace{\onelineskip}
		\noindent
		\textbf{\uppercase{Keywords}}: \keywords
	\end{otherlanguage*}
\end{resumo}
