% ---
% METODOLOGIA
%
% ---

\chapter{Metodologia}

A metodologia adotada para o objeto de estudo foca na observação da tecnologia 
\textit{Java} para o desenvolvimento no conceito \textit{Internet} das Coisas 
através de três diferentes escopos de aplicações. A experimentação das 
aplicações proporcionará um amplo alcance de possibilidades, por meio das 
bibliotecas e ferramentas de \textit{software}.

Os escopos selecionados para a experimentação são:

\begin{itemize}
    
	\item periféricos de baixo nível: apresentar as Interfaces de Programação 
	de Aplicação - \textit{Application Programming Interfaces} (\textit{APIs}) 
	e periféricos destinados a comunicação e interfaceamento de baixo nível com 
	o \textit{Raspberry PI}. A aplicação utilizará o \textit{Java ME} (Edição 
	Micro - \textit{Micro Edition}) com a \textit{Device I/O API} 
	(Entrada/Saída - \textit{Input/Output});
    
	\item arquitetura cliente/servidor: apresentar a utilização de servidores 
	no \textit{Raspberry PI}. A aplicação utilizará o \textit{Java Embedded 
	Suite};
    
	\item processamento de eventos complexos: apresentar o ambiente para 
	desenvolvimento de processamento de eventos complexos. A aplicação 
	utilizará o \textit{Java Oracle Event Processing}.
    
\end{itemize}

As aplicações serão desenvolvidas pelo Ambiente de Desenvolvimento Integrado -  
\textit{Integrated Development Environment} (\textit{IDE}) \textit{Netbeans} 
8.0.2 com o \textit{Kit} de Desenvolvimento de \textit{Software} - 
\textit{Software Development Kit} (\textit{SDK}) \textit{Java} 8. O sistema 
base é composto de um ambiente virtualizado no \textit{Oracle VM VirtualBox} 
com o sistema operacional \textit{Linux Debian}. No caso do \textit{Java ME SDK 
8 Embedded} será utilizado através do sistema operacional \textit{Windows} 7, 
pois não há uma versão para o sistema operacional \textit{Linux}.

As aplicações serão testadas com a plataforma de prototipagem \textit{Raspberry 
PI B+} com o sistema operacional \textit{Raspbian - Debian Wheezy} versão de 
16/02/2015.

A comunicação do computador de desenvolvimento e a plataforma de prototipagem 
será pelo acesso remoto com os utilitários \textit{Secure Shell} (\textit{SSH}) 
e \textit{Virtual Network Computing} (\textit{VNC}) por rede \textit{Ethernet}.

Os resultados serão apresentados através da indicação das principais vantagens 
e desvantagem observadas, dentro do conceito \textit{Internet} das Coisas.
