% ---
% INTRODUÇÃO
%
% ---

\chapter{Introdução}

A \textit{Internet} das Coisas - \textit{Internet of Things} (\textit{IoT}) tem
como objetivo prove a conexão de rede aos objetos e tornando-os inteligentes,
em um sistema composto de um conjunto de três pilares, chamado de \textit{Three
  Is}: Instrumentação, Interconexão e Inteligência \cite{mqttibm2012}.

A Instrumentação captura as informações através de sensores, enquanto a
Interconexão provê o meio para a transferência da informação do ponto de coleta
para o ponto de consumo, finalmente a Inteligência com a responsabilidade sobre
o processamento e análise da informação para derivar o máximo de valor e
conhecimento \cite{mqttibm2012}.

Os anos demonstram o crescimento do número de equipamentos conectados à
\textit{Internet}, em 2000 a quantidade de humanos na Terra foi de 6 bilhões
para 500 milhões de equipamentos, enquanto em 2011 com 7 bilhões de humanos
para 13 bilhões de equipamentos, para 2015 está previsto um número três vezes
maior de equipamentos do que humanos na Terra \cite{mirkopresser2012}.

Os objetos passam a sentir o contexto do ambiente e a comunicar com os humanos,
tornando ferramentas poderosas para a obtenção de conhecimento.

Os exemplos de aplicações são inúmeros, tais como:

\begin{itemize}

    \item Medição do consumo de energia e água, analisando os resultados em
      tempo real, assim avaliando os custos (gastos, créditos e demanda), a
      consciência do desperdício pelo comportamento do consumo, a eficiência
      das instalações (quando realizar melhorias e reparos) e a comparação
      estatística com outras fontes pela mídia social \cite{mirkopresser2012}.

    \item Cuidado contínuo dos pacientes idosos e com doenças crônicas,
      permitindo o repouso em casa ao invés de esperarem em filas e quartos de
      hospitais e clínicas, em que os médicos monitoraram remotamente o estado
      de saúde dos pacientes por equipamentos nas roupas ou distribuídos pela
      casa e/ou ruas \cite{mirkopresser2012}.

    \item Iluminação urbana por regular o controle de luz com a presença de
      carros e pedestres, como também destacar situações adversas como
      vazamento de óleo, acidentes de trânsito e trabalhadores de rua
      \cite{mirkopresser2012}.

\end{itemize}

\newpage
As partes de interconexão e instrumentação são disponibilizadas em uma
variedade de tecnologias como padrão de mercado, diferente da inteligência na
qual requisita o desenvolvimento completo do \textit{software}, desde a fase de
requisitos, análise, desenvolvimento e testes. Dentre as plataformas
disponíveis para o desenvolvimento estão o \textit{Java} e o \textit{C/C++},
juntamente com suas linguagens de \textit{front-end} como o
\textit{Python}. Dos sistemas operacionais estão as distribuições
\textit{Linux} como \textit{Debian} e o \textit{Ångström}, sendo o último
especializado para dispositivos embarcados.

A plataforma selecionada para o estudo é o \textit{Java Embedded}, em que
apresenta compatibilidade com diversas arquiteturas de \textit{hardware}
através de sua Máquina Virtual - \textit{Virtual Machine} (\textit{VM}) e
disponibiliza uma variedade de Interface de Programação de Aplicação -
\textit{Application Programming Interface} (\textit{API}) para periféricos de
baixo nível, arquitetura Cliente/Servidor - \textit{Client/Server} e
Processamento de Eventos Complexos - \textit{Complex Event Processing}
(\textit{CEP}).

A metodologia de trabalho apresentará as tecnologias disponíveis na plataforma
\textit{Java Embedded} e demonstrará exemplos de aplicações com a utilização da
plataforma de prototipagem \textit{Raspberry PI B+}.

\section{Justificativa}

O conceito \textit{IoT} é a próxima evolução da computação, principalmente pela
preocupação com os recursos ambientais, médicos e planejamento urbano através
do processo de monitoração, aquisição e controle dos dados. O mercado
disponibiliza uma ampla variedade de \textit{hardware} para o sensoriamento e
processamento, facilitando a atividade de construir um sistema computacional. O
\textit{software} é o diferencial para a pesquisa e o desenvolvimento de novas
tecnologias, conhecer a plataforma embarcada do \textit{Java} amplia essas
qualidades e direciona para as futuras oportunidades na \textit{IoT}.

\newpage
\section{Objetivos}

\subsection{Geral}

Elaborar um estudo sobre as características e aplicações da plataforma
\textit{Java Embedded} para o domínio de \textit{software} embarcado, focado no
conceito \textit{IoT}.

\subsection{Específicos}

\begin{itemize}

    \item Especificar as características do conceito \textit{IoT}.

    \item Especificar as tecnologias \textit{Java} disponíveis para o
      desenvolvimento no domínio de \textit{software} embarcado.

    \item Apresentar as interfaces para a comunicação de periféricos de baixo
    nível em embarcados.

    \item Apresentar a aplicação da arquitetura Cliente/Servidor em
    embarcados.

    \item Apresentar a aplicação de Processamento de Eventos Complexos em
    embarcados.

\end{itemize}

\section{Metodologia}

Revisão bibliográfica: será feito uma pesquisa sobre o tema do projeto,
procurando informações sobre as tecnologias da plataforma \textit{Java
  Embedded} e a conceitualização da computação ubíqua para o desenvolvimento de
aplicações para o conceito \textit{IoT}.

Caracterização da \textit{Internet} das Coisas: descrever as características da
computação ubíqua (combinação da computação móvel com a computação embarcada)
presentes no conceito \textit{IoT}.

Caracterização do \textit{Java Embedded}: descrever as características,
vantagens e desvantagens das tecnologias da plataforma \textit{Java Embedded}
para o desenvolvimento no domínio de \textit{software} embarcado, utilizando a
plataforma de prototipagem \textit{Raspberry PI B+}.

Primeiro exemplo de aplicação: desenvolver um exemplo de aplicação \textit{Java
  ME} (Edição Micro - \textit{Micro Edition}) com a \textit{Device I/O API}
(Entrada/Saída - \textit{Input/Output}) (Interface de Programação de Aplicação
- \textit{Application Programming Interface}), explanando o interfaceamento de
periféricos de baixo nível presentes em sistemas embarcados.

Segundo exemplo de aplicação: desenvolver um exemplo de aplicação \textit{Java
  Embedded Suite} (\textit{JES}), explanando o uso da arquitetura
Cliente/Servidor em sistemas embarcados.

Terceiro exemplo de aplicação: desenvolver um exemplo de aplicação \textit{Java
  Oracle Event Processing} (\textit{OEP}), explanando a utilização de
Processamento de Eventos Complexos em sistemas embarcados.

Discussão dos resultados: apresentar os resultados obtidos pelos exemplos
aplicados, evidenciando as principais vantagens e desvantagem observadas.
