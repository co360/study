% ---
% ANÁLISE DOS RESULTADOS
%
% ---

\chapter{Análise dos Resultados}

O \textit{Java Embedded} é composto de diversos módulos para o desenvolvimento
de \textit{software} embarcado, compatível com as necessidades encontradas no
conceito de \textit{Internet} das Coisas - \textit{Internet of Things}
(\textit{IoT}).

No \textit{Java ME Embedded} são encontradas diversas ferramentas para o
desenvolvimento, juntamente com uma rica documentação do \textit{Device I/O
  API} (Entrada/Saída - \textit{Input/Output}) (Interface de Programação de
Aplicação - \textit{Application Programming Interface}). Entretanto a
plataforma não é fixa em apenas um sistema operacional, pois existem recursos
voltados para \textit{Windows} e outros para \textit{Linux}. A parte embarcada
é disponibilizada por uma estrutura de projeto com diversos \textit{scripts}
para a execução e/ou interfaceamento com o sistema operacional da aplicação
desenvolvida.

No \textit{Java Embedded Suite} (\textit{JES}) são disponibilizados os
principais módulos para aplicações com servidores através de versões reduzidas
e otimizadas para o ambiente embarcado, no conceito \textit{IoT} faz uso de
tais aplicações por conectividade com outros sistemas e usuários. A plataforma
possui interoperabilidade com demais plataformas do \textit{Java Embedded},
trazendo um grande escopo de fonte de dados. O desenvolvimento é compatível com
o tradicional no Java EE (Edição Empresarial - \textit{Enterprise Edition}),
utilizando um número menor de componentes e sua implantação através de
\textit{scripts}, como no \textit{Java ME Embedded} com a vantagem do
interfaceamento com o sistema operacional.

No \textit{Java Oracle Event Processing} (\textit{OEP}) são disponibilizados um
subconjunto das funcionalidades do \textit{Oracle Event Processing}, otimizados
para o ambiente embarcado. Suas ferramentas e documentações são limitados,
principalmente pelo modelo de programação em Linguagem de Marcação Estendida -
\textit{eXtensible Markup Language} (\textit{XML}), diferente do tradicional
pela linguagem gráfica de Rede de Processamento de Eventos - \textit{Event
    Processing Networks} (\textit{EPNs}). Existe uma dificuldade em sua
utilização para o desenvolvimento em aplicações no conceito \textit{IoT}.
