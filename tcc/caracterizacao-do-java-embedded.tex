% ---
% CARACTERIZAÇÃO DO JAVA EMBEDDED
%
% ---

\chapter{Caracterização do Java Embedded}

O \textit{Java Embedded} oferece aos desenvolvedores o estado da arte de uma
plataforma otimizada para um espectro de dispositivos embarcados, envolvendo
bibliotecas e ferramentas para o melhor aproveitamento dos Sistema-em-um-chip -
\textit{System-on-a-chip} (\textit{SoC}) através de um conjunto de plataformas,
conforme o \textit{footprint} da aplicação a ser desenvolvida.

A plataforma possui componentes 100\% em código \textit{Java}, desenvolvendo
aplicações embarcadas mais simples do que em plataforma \textit{C}, integrando
ferramentas de desenvolvimento, tais como \textit{NetBeans} e
\textit{Eclipse}. Os \textit{bytecodes} compilados para \textit{desktop} são
identicos aos executados no embarcado, com exceção da velocidade de execução,
permitindo as etapas de desevolvimento como codificação, execução,
\textit{debug} e \textit{profile} sem a necessidade do dispositivo embarcado.

\section{Plataforma Java Embedded}

As plataformas abordadas no respectivo trabalho serão: \textit{Java ME
  Embedded}, \textit{Java Embedded Suite} (\textit{JES}) e \textit{Java Oracle
  Event Processing} (\textit{OEP}).

\subsection{Java ME Embedded}

O \textit{Java ME Embedded} é uma versão reestruturada da plataforma
\textit{Java ME} com a inclusão da \textit{Device I/O API} (Entrada/Saída -
\textit{Input/Output}) (Interface de Programação de Aplicação -
\textit{Application Programming Interface}), destinadas aos periféricos de
baixo nível encontrados nos dispositivos embarcados, tais como Entrada/Saída de
Propósito Geral - \textit{General Purpose Input/Output} (\textit{GPIO}),
Conversor Analógico para Digital - \textit{Analog to Digital Converter}
(\textit{ADC}), Circuito Inter-Integrado - \textit{Inter-Integrated Circuit}
(\textit{I2C}), Interface Periférica Serial - \textit{Serial Peripheral
  Interface} (\textit{SPI}), Transmissor/Receptor Assíncrono Universal -
\textit{Universal Asynchronous Receiver/Transmitter} (\textit{UART}) e etc.

A plataforma \textit{Java ME Embedded} consiste de duas versões:

\begin{itemize}

    \item \textit{Java ME Embedded}: dedicado para as aplicações sempre ativa
    (\textit{always-on}), sem interface gráfica com o usuário
    (\textit{headless}) e com conexões a dispositivos. Possui
    \textit{footprint} menor que 1 \textit{megabyte} de memória;

    \item \textit{Java ME Embedded Client}: dedicado para as aplicações com
    maior necessidade de características da plataforma \textit{Java}. Possui
    \textit{footprint} menor que 10 \textit{megabytes} de memória.

\end{itemize}

Exemplos de aplicações:

\begin{itemize}

    \item prototipagem para aplicações de computação embarcada;

    \item possibilidades para utilização de sensores como fonte de dados em
    aplicações para o conceito \textit{Internet} das Coisas -
    \textit{Internet of Things} (\textit{IoT}).

\end{itemize}

\subsection{Java Embedded Suite}

O \textit{JES} é uma versão da plataforma \textit{Java EE} (Edição Empresarial
- \textit{Enterprise Edition}) para os sistemas embarcados, trazendo a
capacidade da arquitetura Cliente/Servidor - \textit{Client/Server} através de
banco de dados - \textit{databases}, serviços \textit{Web} - \textit{Web
  services} e aplicações \textit{Servlet}.

Os componentes disponíveis para a plataforma são:

\begin{itemize}

    \item \textit{GlassFish}: versão reduzida do servidor de aplicações
    \textit{GlassFish}, com suporte apenas para os componentes \textit{Servlet
    3.0} e \textit{Bean Validation 1.0};

    \item \textit{Java DB}: versão otimizada do banco de dados \textit{Derby}
    para uso embarcado (não possui características para arquitetura
    Cliente/Servidor), acessado via Linguagem de Consulta Estruturada -
    \textit{Structured Query Language} (\textit{SQL}) sobre o Conectividade de
    Banco de Dados \textit{Java} - \textit{Java Database Connectivity}
    (\textit{JDBC});

    \item \textit{Jersey}: \textit{framework} para serviços \textit{RESTful Web
    Services}, conforme a especificação \textit{JAX-RS} (\textit{JSR 311}).

\end{itemize}

\subsection{Java Oracle Event Processing}

O \textit{OEP} é uma versão do sistema de Processamento de Eventos Complexos -
\textit{Complex Event Processing} (\textit{CEP}) para os sistemas embarcados,
trazendo a capacidade de transformar dados em informação para os dispositivos.
O servidor de \textit{OEP} executa sobre o suporte do \textit{JES}, com um
conjunto reduzido de características e ferramentas.

\section{Vantagens e Desvantagens}

A plataforma \textit{Java Embedded} possui vantagens e desvantagens em relação
a plataforma nativa (\textit{C}).

\newpage
\subsection{Vantagens}

\begin{itemize}

    \item Suporte a aplicações \textit{headless}: serviços executados em
    segundo plano;

    \item Segurança \textit{sandbox}: a máquina virtual constrói um ambiente de
    execução para a aplicação, diferentemente da plataforma nativa em que não
    possui a característica;

    \item Múltiplos processos: capacidade para execução multitarefa de
    aplicações;

    \item Comunidade: a tecnologia \textit{Java} possui uma grande comunidade
    de desenvolvedores em todo mundo, ideal para o aprendizado e suporte, na
    plataforma nativa a comunidade é fragmentada;

    \item Escalabilidade: excelente capacidade de gerenciamento de recursos e
    adição de novas funcionalidades a aplicação;

    \item Capacidade de atualização: a execução em máquina virtual traz para o
    sistema possui um excelente mecanismo para atualização da aplicação, na
    plataforma nativa o sistema é inferior em dificuldade.

\end{itemize}

\subsection{Desvantagens}

A principal desvantagem em relação a plataforma nativa está no desempenho, pois
a máquina virtual utiliza de recursos como memória e processamento, além dos
requeridos para a aplicação.

Uma abordagem para minimizar os problemas está em definir \textit{profiles}
para o tipo de aplicação, assim reduzindo a carga da máquina virtual.

\section{Otimização da JVM}

A Máquina Virtual Java - \textit{Java Virtual Machine} (\textit{JVM}) é
otimizada para as Unidades Central de Processamento - \textit{Central
  Processing Units} (\textit{CPU}) da arquitetura de Máquina \textit{RISC} 
  Avançada - \textit{Advanced RISC Machine} (\textit{ARM}), com suporte a
\textit{hard-float} no \textit{Kit} de Desenvolvimento \textit{Java} -
\textit{Java Development Kit} (\textit{JDK}) 1.8. Várias plataformas de
\textit{hardware} são compatíveis como: \textit{ARM}, \textit{Intel},
\textit{Atmel}, dentro outros.
