% ---
% EVOLUÇÃO DA COMPUTAÇÃO
%
% ---

\chapter{Evolução da Computação}

A computação passou por diversas evoluções no último século, devido as
pesquisas e desenvolvimento em diferentes áreas de estudo, seus resultados veem
através de novas tecnologias e aplicações, como o objeto de estudo: a aplicação
\textit{Internet} das Coisas - \textit{Internet of Things} (\textit{IoT}) e a
tecnologia \textit{Java Embedded}.

\section{Paradigmas Computacionais}

Os paradigmas computacionais delimitam as principais evoluções da computação,
sendo os principais marcos:

\begin{itemize}

    \item sistemas \textit{mainframe}: na década de 60, consistindo de grandes
    computadores do tamanho de salas inteiras em departamentos de processamento
    de dados. Suas principais características estão: a grande capacidade de
    Entrada/Saída - \textit{Input/Output} (\textit{I/O}), o relacionamento de
    uma máquina para um usuário e o processamento em lote (\textit{batch});

    \item sistemas por terminais (\textit{terminals}): na década de 70,
    consistindo de dispositivos remotos conectados ao computador central
    (\textit{mainframe}) via rede telefônica para a entrada e saída de dados
    aos usuários. Suas principais características estão: o processamento
    interativo e multitarefa, o relacionamento de uma máquina para muitos
    usuários e o compartilhamento de tempo de processamento (\textit{time
        sharing});

    \item computadores pessoais (\textit{desktop}): na década de 80,
    consistindo de pequenos computadores com preço acessível para uso em
    pequenas aplicações como escritório até grandes aplicações como jogos,
    sendo estendido para aplicações em ambientes de sistemas embarcados. Suas
    principais características estão: processamento local e interativo e o
    relacionamento de uma máquina para um usuário;

    \item redes de computadores (\textit{network}): na década de 90,
    consistindo de interconexão de computadores com as vantagens de
    comunicação, compartilhamento de recursos e acesso não local. Sua
    principais características estão: a implementação do modelo de arquitetura
    Cliente/Servidor - \textit{Client/Server} e a comunicação centralizada; e

    \item sistemas distribuídos (\textit{distributed systems}): a partir da
    década de 90, consistindo de um conjunto de computadores compartilhando o
    processamento para a aplicação, aparentando com um único computador ao
    usuário. Suas principais características estão: a implementação do modelo
    de arquitetura Distribuído, a comunicação descentralizada e as aplicações
    por \textit{Grid} e \textit{Cloud}.

\end{itemize}

O próximo candidato de paradigma computacional é a computação ubíqua, uma união
das pesquisas e desenvolvimento entre computação móvel e computação pervasiva.

A computação móvel traz aos usuários a informação contínua, independente de sua
localização física, enquanto a computação pervasiva traz o processamento
autônomo para o usuário em diversos dispositivos, adicionando inteligência ao
mesmo.

\section{Computação Ubíqua}

A computação ubíqua tem origem nos trabalhos de \textit{Mark Weiser} da
\textit{Xerox Palo Alto Research Center} (\textit{PARC}) durante o início da
década de 90, com a visão da criação de ambientes inteligentes através de
sistemas embarcados (computação pervasiva) aplicados aos objetos do dia a dia,
trocando informações entre si e pela rede de computadores. Segundo
\textit{Weiser} \cite[p. ~94]{markweiser1991}: \quotes{Especializados elementos
    de \textit{hardware} e \textit{software}, conectados por fios, ondas de
    rádio e infravermelho, serão tão ubíquos que ninguém irá notar sua
    presença.}

Nesse ambiente o usuário estaria cercado por diversos dispositivos realizando
computação imperceptivelmente, sem uma interface de usuário como em
computadores pessoais (menos intrusão). Esses computadores operam autonomamente
e realizam tomadas de decisões e interações inteligentes, conforme o contexto
do ambiente.

A computação ubíqua traz aos desenvolvedores um alto grau de mobilidade e
imersão computacional. O grau de mobilidade refere-se a capacidade de locomoção
física dos serviços computacionais, enquanto o grau de imersão computacional
refere-se a capacidade de inteligencia do dispositivo em detectar, explorar e
construir de maneira dinâmica o contexto do ambiente.

\section{Internet das Coisas}

O termo \textit{Internet} das Coisas ou \textit{Internet of Things}
(\textit{IoT}) surgiu em 1999 pelo pesquisador britânico \textit{Kevin Ashton}
do Instituto de Tecnologia de \textit{Massachusetts} - \textit{Massachusetts
    Institute of Technology} (\textit{MIT}) durante uma apresentação executiva.

A ideia era utilizar as tecnologias de Identificação por Rádio Frequência -
\textit{Radio-Frequency IDentification} (\textit{RFID}) e sensores para
habilitarem os computadores para observar, identificar e compreender o mundo,
sem a limitação dos dados inseridos por usuários \cite{kevinashton2009}.

\newpage
A definição do \textit{Cluster} Europeu de Pesquisa da \textit{Internet} das
Coisas - \textit{European Research Cluster on the Internet of Things} -
(\textit{IERC}) \cite[p. ~26]{iangsmith2012} para \textit{IoT} é:

\begin{citacao}
    Infraestrutura de rede dinâmica global com capacidades de autoconfiguração
    com base em protocolos de comunicação padronizados e interoperáveis onde as
    coisas físicas e virtuais têm identidades, atributos físicos,
    personalidades virtuais, utilizam interfaces inteligentes e estão
    perfeitamente integrados na rede de informação.
\end{citacao}

Hoje o termo \textit{IoT} é utilizado para relacionar comercialmente a
Computação Ubíqua.

\section{A Tecnologia Java}

A tecnologia \textit{Java} foi desenvolvida por um grupo de engenheiros da
\textit{Sun} (hoje adquerida pela \textit{Oracle}) conhecidos por
\quotes{\textit{Green Team}} em 1991, com o propósito de contribuir para os
dispositivos digitais para serem inteligentes, funcional e de um meio mais
divertido.

Liderado por \textit{James Gosling}, o time desenvolveu a linguagem de
programação \textit{Java} e demonstrou através de um dispositivo portátil para
a indústria de televisão a cabo, como também incorporou ao navegador de
\textit{Internet Netscape Navigator} em 1995, após a explosão da \textit{Web}.

Hoje a tecnologia \textit{Java} possui um amplo de escopo de aplicações, sendo
de \textit{chips} Sistema Global para Comunicações Móveis - \textit{Global
    System for Mobile Communications} (\textit{GSM}) até grandes servidores
empresariais; e com interesses na \textit{IoT}, a \textit{Oracle} lançou a
plataforma \textit{Java Embedded}.
