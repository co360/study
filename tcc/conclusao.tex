\chapter{Conclusão}

O conceito \textit{Internet} das Coisas alcança cada vez mais as aplicações do 
dia a dia, trazendo novas possibilidades e desafios para o mundo da tecnologia.
A tecnologia \textit{Java} incorpora diversas ferramentas para o 
desenvolvimento de sistemas embarcados através da plataforma \textit{Java 
Embedded}.

Diversos escopos são preenchidos, de periféricos de baixo nível até 
processamento de eventos complexos, porém prover a característica de 
produtividade não é o único ponto de escolha, o desempenho ainda é a principal 
qualidade visada principalmente em dispositivos com baixo poder computacional.
O Java permite o meio termo entre desempenho e produtividade, principalmente 
pela diversidade de ferramentas e Interfaces de Programação de Aplicação - 
\textit{Application Programming Interfaces} (\textit{APIs}), como também 
versões otimizadas da Máquina Virtual Java - \textit{Java Virtual Machine} 
(\textit{JVM}).

Aspectos importantes a serem previstos estão na especificação e desenvolvimento 
de protocolos, \textit{frameworks} e sistemas operacionais especializados.
