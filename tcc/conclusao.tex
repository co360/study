% ---
% CONCLUSÃO
%
% ---

\chapter{Conclusão}

O conceito \textit{Internet} das Coisas - \textit{Internet of Things}
(\textit{IoT}) alcança cada vez mais as aplicações do dia a dia, trazendo novas
possibilidades e desafios para o mundo da tecnologia.  A tecnologia
\textit{Java} incorpora diversas ferramentas para o desenvolvimento de sistemas
embarcados através da plataforma \textit{Java Embedded}.

As aplicações realizadas no estudo demonstram um diferente grau de maturidade
entre as plataformas, conforme o nível de abstração empregada.  No \textit{Java
    ME Embedded} são disponibilizados um ambiente completo de desenvolvimento
(emulador, \textit{profile}, \textit{debug}...), assim como sua integração com
o \textit{Java Embedded Suite} na forma de uma fonte de dados.  No \textit{Java
    Embedded Suite} são disponibilizados diversos componentes para a
implantação de um servidor em embarcados, não são completos como a versão
\textit{desktop}, porém traz novos horizontes com a utilização de \textit{Web},
banco de dados e serviços \textit{Web} em aplicações embarcadas.  No
\textit{Java Oracle Event Processing} ainda falta um melhor desenvolvimento do
ambiente, os \textit{plugins} para Ambiente de Desenvolvimento Integrado -
\textit{Integrated Development Environment} (\textit{IDE}) estão desatualizados
e sua documentação não aproveita o escopo de aplicações embarcadas.

No geral diversos escopos são preenchidos, de periféricos de baixo nível até
processamento de eventos complexos, porém prover a característica de
produtividade não é o único ponto de escolha, o desempenho ainda é a principal
qualidade visada principalmente em dispositivos com baixo poder computacional.
O \textit{Java} permite o meio termo entre desempenho e produtividade,
principalmente pela diversidade de ferramentas e Interfaces de Programação de
Aplicação - \textit{Application Programming Interfaces} (\textit{APIs}), como
também versões otimizadas da Máquina Virtual Java - \textit{Java Virtual
    Machine} (\textit{JVM}).

Aspectos importantes a serem previstos estão na especificação e desenvolvimento
de protocolos, \textit{frameworks} e sistemas operacionais especializados.
